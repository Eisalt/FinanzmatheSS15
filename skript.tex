\documentclass[]{scrartcl}
\usepackage{hyperref}
\usepackage[activate={true,nocompatibility},final,tracking=true,kerning=true,spacing=true,factor=1100,stretch=10,shrink=10]{microtype}
\usepackage{amssymb, amsmath, amsthm}
\usepackage{dsfont}
\usepackage[utf8]{inputenc}
\usepackage[T1]{fontenc}
\usepackage[ngerman]{babel}
\usepackage{nicefrac}
\usepackage{enumitem}
%\usepackage{paralist} %\begin{compactenum}[(i)] führt zu (i), (ii)
\usepackage{lmodern}
%\usepackage[garamond]{mathdesign}
\usepackage[utopia]{mathdesign}


\numberwithin{equation}{section}


\theoremstyle{plain}
\newtheorem{satz}{Satz}[section]
\newtheorem{lem}[satz]{Lemma}
\newtheorem{kor}[satz]{Korollar}
\newtheorem{prop}[satz]{Proposition}
\theoremstyle{definition}
\newtheorem{defi}[satz]{Definition}
\theoremstyle{remark}
\newtheorem{bem}[satz]{Bemerkung}
\newtheorem*{bemerkung}{Bemerkung}
\newtheorem{bsp}[satz]{Beispiel}
\theoremstyle{proof}
\newtheorem*{bew}{Beweis}
\makeatletter
\g@addto@macro{\thm@space@setup}{\thm@headpunct{}}
\makeatother
\newtheorem*{frag}{Fragestellung:}

\newcommand{\abs}[1]{\lvert #1 \rvert}
\newcommand{\Abs}[1]{\lVert #1 \rVert}

\newcommand{\N}{\mathbb{N}}
\renewcommand{\wr}{(\Omega,\mathcal{F},\mathbb{P})}
\renewcommand{\P}{\mathbb{P}}
\newcommand{\F}{\mathcal{F}}
\newcommand{\R}{\mathbb{R}}
\newcommand{\E}{\mathbb{E}}
\renewcommand{\d}{\operatorname{d}}
\newcommand{\call}{\operatorname{Call}}
\renewcommand{\put}{\operatorname{Put}}
\renewcommand{\L}{\operatorname{L}}
\newcommand{\Ss}{\nicefrac{S_0^0}{S_1^0}}

\newcommand{\epm}{Einperiodenmodell}
\newcommand{\am}{Arbitragemöglichkeit}
\newcommand{\af}{arbitragefrei}
\newcommand{\amm}{Äquivalentes Martingal Maß}

\title{Finanzmathematik in diskreter Zeit}
\begin{document}
	\addtocounter{section}{-1}
	\section{Modellbildung und Einführung}
	Modelliert werden sollen Handelsstrategien für Güter  auf einem Markt, z.B. Staatsanleihen, Aktien, \ldots
	\begin{itemize}
		\item das j-te Handelsgut  zur Zeit $t\geq 0$ hat Preis $S_t^j$
		\item ein Händler hat zur Zeit $t\geq 0$ die Stückzahl $x_t^j\ (\phi_t^j)$ des j-ten Handelsgutes
		\item der Wert des Portfolios zum Zeitpunkt $t\geq 0$ ist \[ x_t\cdot S_t=\sum_{j=1}^nx_t^jS_t^j\]
		\item die Preise $S_t^j$ hängen vom Zufall ab, die Stückzahl $x_t^j$ kann der Händeler durch Kauf und Verkauf beeinflussen
		\item $x_t^j$ muss nicht aus $\N$ sein; $t\mapsto\phi_t$ heißt Handelsstrategie
	\end{itemize}
	\begin{frag}\leavevmode
		\begin{enumerate}
			\item mit welcher Handelsstrategie werde ich sicher Reich?
			\item Was ist der faire Preis einer Option?
		\end{enumerate}
	\end{frag}
	\section{Das Einperiodenmodell}
	Es gibt nur einen Hadelszeitpunkt $t=0$ und einen Abrechungszeitpunkt $t=1$.
	\begin{defi}
		Das \epm\ mit $d+1$ Basisgütern besteht aus 
		\begin{itemize}
			\item einem W-Raum $(\Omega,\mathcal{F},\mathbb{P})$
			\item einem Preisvektor $S_0=(S_0^0,S_0^1,\ldots,S_0^d)\in\R^{d+1},\ S_0> 0$
			\item ZVen  $S_1^1,\ldots,S_d^1:\Omega\to\R$ und einem nicht zufälligen $S_1^0> 0$, $S_t^0\ t\in\{0,1\}$ heißt Num\'eraire und ist risikolos
			\item die Zahl $r> -1$ mit $S_1^0=(1+r)S_0^0$ heißt Zinssatz
			\item das diskontierte \epm\ ergibt nich, wenn man $(S_t^0,\ldots,S_t^d)$ durch 
			\[(\tilde{S}_0^d,\ldots,\tilde{S}_t^d):=(\nicefrac{S_t^0}{S_t^0},\ldots,\nicefrac{S_t^d}{S_t^0})\text{ für }t\in\{0,1\} \] ersetzt 
		\end{itemize}
	\end{defi}
	\begin{bem}
		$S_t^j<0$ ist meistens erlaubt, für $j\ge1$.
	\end{bem}
	\begin{defi}
		Ein Portfolio im \epm\ ist ein Vektor  $x=(x^0,x^1,…,x^d)\in \R^{d+1}$ der Stückzahlen. $V_t(x)=x\cdot S_t$ ist der Wert des Portfolios zum Zeitpunkt $t\in\{0,1\}$.
	\end{defi}
	\begin{bem}\leavevmode
		\begin{itemize}
			\item $x^j\le0$ ist erlaubt
			\item $x^0<0$ ist eine Kreditaufnahme in Höhe von $\abs{x^0}S^0$, \\Rückzahlung ist $S_1^0\abs{x^0}=(1+r)S^0_0\abs{x^0}$
			\item $x^j<0$ ist ein Leerverkauf und muss mit $x^jS_1^j$ glattgestellt werden (short position)
			\item $x^j>0$ long position
		\end{itemize}
	\end{bem}
	\begin{defi}
		Ein Portfolio $x\in\R^{d+1}$ im \epm\ heißt \am, falls gilt:
		\begin{itemize}
			\item $V_0(x)\equiv x\cdot S_0\le0$
			\item $V_1(x)\equiv x\cdot S_1\ge 0$
			\item $\P (x\cdot S_1>0)>0$
		\end{itemize}
	\end{defi}
	Eine Grundannahme ist die Arbitragefreiheit, dies ist realistisch, da eine Arbitrage durch Ausnutzung verschwindet.
	\begin{lem}
		Sei $S_t$ ein \epm und $\tilde{S}_t$ das diskontierte \epm
		\begin{enumerate}
			\item sei $x$ ein Portfolio. Dann gilt \[x\text{ ist \am\ für }S_t\Longleftrightarrow x\text{ ist \am\ für }\tilde{S}_t\]
			\item es sind äquivalent:
			\begin{enumerate}
				\item es existiert eine \am\ für $S_t$
				\item es existiert eine \am\ für $\tilde{S}_t$
				\item es existiert ein Portfolio $y\in\R^{d+1},\ c\in \R$ mit 
				\begin{align}\label{1.1}
					y\tilde{S}_0=c,\ y\tilde{S}_1\ge c\ \P f.s.,\ 
				\end{align}
				\item für alle $c\in\R$ gilt, es existiert ein $y\in\R^{d+1}$, so dass \ref{1.1} gilt
			\end{enumerate}
		\end{enumerate}
	\end{lem}
	\begin{bew}\leavevmode
		\begin{enumerate}
			\item klar, teil durch $S_0^0$
			\item
			\begin{itemize}[leftmargin=4.5em]
				\item[(a) $\Leftrightarrow$ (b)] klar wegen (a)
				\item[(b)\ \ $\Rightarrow$ (d)] sei $x$ ein \am\ und $c\in\R$. Setze
				\begin{align*}
					y^i&=
					\begin{cases}
					x^i &\text{für }i\ge 1\\
					x^0+c-\tilde{V}_0(x) &\text{für } x=0 
					\end{cases}\\
					\Longrightarrow y\tilde S_0&=x\tilde S_0+(c-\tilde V_0(x))\tilde S_0^0=c
					\intertext{und}
					y\tilde{S}_1&=x\tilde S_1+(c-\tilde V_0(x))\tilde S_1^1\\
					&=\tilde V_1(x)-\tilde V_-(x)+c\ge 0\ \P f.s.,\ \P (yS_1>0)>0\text{ wegen (b)}
				\end{align*}
				\item[(d)\ \ $\Rightarrow$ (c)] klar
				\item[(c)\ \ $\Rightarrow$ (b)] erfülle nun $y$ (b). Setze
				\begin{align*}
					x^i&=
					\begin{cases}
					y^i&\text{für }i\ge 1\\
					y^0-c&\text{für }i=0
					\end{cases}\\
					\intertext{Dann}
					\tilde V_0(x)&=\tilde V_0(y)-c\tilde S_0^0=c-c=0
					\intertext{und}
					\tilde V_1(x)&=\tilde V_1(y)-c\tilde S_1^0\ge 0\ \P f.s.\text{ und }\ \P(\tilde V_1(x)>0)=\P(\tilde V_1(y)>c)>0
				\end{align*}
			\end{itemize}
		\end{enumerate}
		\qed
	\end{bew}
	\section{Der 1. Hauptsatz der Preistheorie im \epm}
	\begin{defi}
		Sei $(\Omega,\F)$ ein Messraum, $\mu$, $\nu$ Maße auf $(\Omega,\F)$. Man nennt $\mu$ absolut stetig bezüglich $\nu$ (schreibe $\mu\ll\nu$), falls gilt: Jede $\nu$-Nullmenge ist auch $\mu$-Nullmenge, d.h.\begin{align*}
			\forall A\in\F\ \nu(A)=0\Rightarrow\mu(A)=0
		\end{align*}
		$\nu$ und $\mu$ sind äquivalent, falls $\nu\ll\mu$ und $\mu\ll\nu$.
	\end{defi}
	\begin{defi}
		Seien $\mu$, $\nu$ Maße und $f\colon\Omega\to\R^+$ messbar. Man sagt, $\mu$ hat Dichte $f$ bezüglich $\nu$ wenn gilt:
		\begin{align*}
			\mu(A)=\int_Af(\omega)\nu(\operatorname{d}\omega),\ \forall A\in\F 
		\end{align*}
		schreibe $\d\mu=f\d\nu$ oder $f=\nicefrac{\d\mu}{\d\nu}$ oder $\mu=f\nu$.
	\end{defi}
	\begin{lem}
		Seien $f$ und $g$ zwei Dichten von $\mu$ bezüglich $\nu$. Dann ist $f=g\ \nu$f.s. Falls also ein Dichte existiert, ist sie auch fast sicher eindeutig.
	\end{lem}
	\begin{bew}
		Nach Vorraussetzung ist 
		\begin{align*}
			&\int_Af\d\nu=\mu(A)=\int_Ag\d \nu\\
			\Longrightarrow&\int_A(fg)1_A\d\nu=0\ \forall A\in\F\\
			\intertext{Mit}
			A&=\{\omega\colon f(\omega)\ge g(\omega)\}\\
			\intertext{folgt}
			0&=\int(f-g)1_A\d\nu\\
			\Longrightarrow&(f-g)1_{(f\ge g)}=0\ \nu f.s.
		\end{align*}
		Ebenso:
		\begin{align*}
			(f-g)1_{\{f<g \}}=0\ \nu f.s. \Longrightarrow \text{ Behauptung}
		\end{align*}
		\qed
	\end{bew}
	\begin{lem}
		Falls $\mu=f\nu$, dann gilt $\mu\ll\nu$
	\end{lem}
	\begin{bemerkung}
		Die Umkehrung gilt auch, falls $\mu$, $\nu$  $\sigma$-endlich sind.
	\end{bemerkung}
	\begin{defi}
		Ein Maß $\mu$ heißt $\sigma$-endlich, wenn es $(A_n)_{n\in\N}\subset\F$ mit $\mu(A_n)<\infty$ und $\bigcup_nA_n=\Omega$ gibt.
	\end{defi}
	\begin{satz}[Radon-Nikodym]
		Seien $\mu$ und $\nu$ zwei $\sigma$-endliche Maße, dann sind Äquivalent:
		\begin{itemize}
			\item $\mu$ hat ein Dichte bezüglich $\nu$
			\item $\mu\ll\nu$
		\end{itemize}
	\end{satz}
	\begin{prop}
		Seien $\mu\ll\nu$, Dichte $f$, $\mu$ und $\nu$ $\sigma$-endlich. Dann gilt:
		\begin{align*}
			\mu\approx\nu\Longleftrightarrow f>0 \ \nu f.s.
		\end{align*}
		Falls $\mu\approx\nu$, dann ist $g=\nicefrac{1}{f}$ die Dichte von $\nu$ bezüglich $\mu$.
	\end{prop}
	\begin{bew}
		$\Longleftarrow$:\\ Sei $f>0\ \nu f.s.$. Setze 
		\begin{align*}
			g=\mathds{1}_{\{f>0 \}}\nicefrac{1}{f}\Longrightarrow g=\nicefrac{1}{f}\ \nu f.s.
		\end{align*}
		Setze $\tilde{\nu}=g\mu$. Dann ist 
		\begin{align*}
			\tilde{\nu}=g\mu=gf\nu&\Longrightarrow\tilde{\nu}(A)=\int_Agf\d\nu=\int_A\mathds 1_{\{f>0 \}}\d\nu\qquad \forall A\in\F\\
			\Longrightarrow\nu=\tilde{\nu}&\Longrightarrow\nu\ll\mu\text{ Dichte }\nicefrac{1}{f}
		\end{align*}
		$\Longrightarrow$:\\ Sei $\mu\approx\nu$ und Satz 2.6. Setze $\nu=g\mu$. Dann:
		\begin{align*}
			\nu&=g\mu=gf
			\intertext{ also }
			\int_A1\d\nu&=\int_Agf\d\nu\qquad\forall A\in\F\\
			\Longrightarrow gf&=1\ \nu f.s.\longrightarrow f>0\ \nu f.s. \text{ und } g=\nicefrac{1}{f}\ \nu f.s.
		\end{align*}
	\end{bew}
	\section{Preisbestimmung von Kontrakten im \epm}
	\setcounter{satz}{14}
	\begin{defi}
		Die Intervallgrenzen $\pi_{\inf}(\xi):=\inf\Pi(\xi)$ und $\pi_{\sup}(\xi):=\sup\Pi(\xi)$ heißen Arbitragegrenzen für $\xi$.
	\end{defi}
	\begin{bem}
		Bestimmung von Arbitragegrenzen ist oft schwierig, denn man muss in $\tilde{\Pi}$ alle \amm e kennen. Falls man $\P$ nicht vorher festlegt, werden diese Grenzen trivial. 
	\end{bem}
	\begin{bsp}
		Sei $(\Omega,\F,\P)$ ein \af\ \epm\ $(S^0,…,S^d)$ und $r=0$. Sei 
		\begin{align*}
			s_i^j\ge 0\ P\ f.s.\text{ und }S^0_1=(1+r)S_0^0
		\end{align*}
		Für $K>0$ sei 
		\begin{align*} 
			\call(K,1,i)(\omega)=(S_1^i(\omega)-K)_+ \end{align*} 
			Wegen $S_1^i\ge$ ist 
		\begin{align*}
			\call(K,1,i)&\le S_1^i\ \P f.s.\\
			\intertext{daraus folgt, dass für alle Äquivalenten Martingalmaße $\mu$ gilt} 
			\mu:\E_{\mu}(\call(K,1,i))&\le\E_{\mu}(S_1^i)=S_0^i\\ 
			\Longrightarrow\pi_{\sup}(\call(K,1,i))&\le S_0^i 
		\end{align*} 
		das heißt eine Call Option darf nie mehr kosten als das Underlying.\\
		Für Schranken an $\pi_{inf}(\call(K,1,i))$ gilt: 
		\begin{align*}
			\E_{\mu}(\call(K,1,i))=\E_{\mu}((S_1^i-K)_+)\overset{\footnote{Jensen Ungleichung}}{\ge} (\E_{\mu}(S_1^i)-K)_+=(S_0^i-K)_+ 
		\end{align*} 
		eine Call Option darf nie billiger sein als Aktienpreis minus abdiskontiertem Strike-Preis.\\
		Put-Option:\\
		\begin{align*}
			(\nicefrac{K}{1+r}-S_0^i)\le\pi_{\inf}(\put(K,i,1))\le\pi_{\sup}(\put(K,i,1))\le\nicefrac{K}{1+r}
		 \end{align*}
 	\end{bsp}
 	\begin{satz}
 		$(S^1,…,S^{d+1})$ sei ein \epm, Kontrakte $\xi^1,…,\xi^n\ n\ge2$ . Die Preise $\pi_i=\pi(\xi^i)$ seien so gewählt, dass das Modell $(S^0,…,S^d,S^{d+1},…,S^{d+n})$ mit $S^{d+i}_0=\pi_i, S^{d+i}_1=\xi^i$ \af\ ist. Dann gilt:
 		\begin{itemize}
 			 	\item Falls $\xi^i\le\xi^j\ \P f.s.$, dann ist auch $\pi_i\le\pi_j\ \P f.s.$
 			 	\item Der Kontrakt $(\xi^i\pm\xi^j) $ ist replizierbar in $(S^0,…,S^{d+n})$ mit $\hat{\pi}(\xi^i\pm\xi^j)=\pi(\xi^i)\pm\pi(\xi^j)$ 
 		\end{itemize}
 	\end{satz}
 	\begin{bew}\leavevmode
 		\begin{itemize}
 			\item[b)] Da $(S^0,…,S^{d+n})$ \af \ ist, existiert ein \amm\ $\mu$ 
 			\begin{align*}
	 			\Longrightarrow\pi(\xi^i)=S^i_0=\nicefrac{S^0_0}{S^0_1}\E_{\mu}(S_1^1)=\nicefrac{S^0_0}{S^0_1}\E_{\mu}(\xi^i)\ \forall i\le n 
	 		\end{align*} 
	 		und \begin{align*}
		 		\hat{\pi}(\xi^i\pm\xi^j)=(3.1bzw3.7)=\nicefrac{S_0^0}{S_1^0}\E_{\mu}(\xi^i\pm\xi^j)=\Pi(\xi^i)\pm\Pi(\xi^j)=\Pi_i\pm\Pi_j 
		 	\end{align*} 
		 	denn $\xi^i\pm\xi^j$ ist replizierbar mit Portfolio $x=(0,…,0,1d+i,0,…,0,\pm1d+j,0,…,0)$ $\Longrightarrow b$
 			\item[a)] 
	 		\begin{align*}
		 		\Pi(\xi^j)-\Pi(\xi^i)=(b) \hat{\Pi}(\xi^j-\xi^i)=\nicefrac{S_0^0}{S_1^0}\E_{\mu}(\xi^j-\xi^i)\ge 0 
		 	\end{align*}
		 	\qed
 		\end{itemize}
 	\end{bew}
 	\begin{kor}
 		Im \af en  \epm \ sei $\pi(\call(K,1,i))$ ein begiebiger \af er  Preis für $\call(K,1,i)$. Dann gilt
 		\begin{align*}
		 	\pi(\call (K,1,i))=S_0^i-\nicefrac{S_0^0}{S_1^0}K+\pi(\put (K,1,i))\qquad\text{Put-Call Parität} 
		 \end{align*} 
		 das heißt ein eimal festgelegter Peis von Call bestimmt den Peis von Put und umgekehrt
 	\end{kor}
 	\begin{bew}
 		\begin{align*}
	 		\put(K,1,i)=(K-S_1^i)_+
		 \end{align*}
		 \begin{align*}
			 \call(K,1,i)=(S^i_1-K)_+=(K-S_1^i)_++S_1^i-K=\put(K,1,i)+S_1^i-K 
		\end{align*} 
 		Sei $\mu(\call)$ ein  \af er Preis. Dann folgt mit 3.18b und $\xi^1=S^i_1,\ \xi^2=\call$ das \begin{align*}
	 		\Pi(\put(K,1,i))=\Pi(\call(K,1,i))-\pi(S^i_1)+\pi(K)
	 	\end{align*} 
	 	mit $\pi(K)=\nicefrac{S_0^0}{S_1^0}K$. \qed
 	\end{bew}
 	\section{Der 2. Hauptsatz der Preistheorie im \epm}
 	Wir wissen:
 	\begin{itemize}
 		\item Wenn der Markt vollständig ist, dann ist nach def 3.11 jeder Kontrakt replizierbar und dies ist äquivalent dazu, dass es für jeden Kontrakt genau einen \af en Preis gibt
 		\item Für $\xi\in\mathcal{H}$ und für jedes Martingalmaß $\mu$ ist $\hat{\Pi}(\xi)=\nicefrac{S_0^0}{S_1^0}\E_{\mu}(\xi)$ nach satz 3.7
 		\item Die Menge aller \af \ Preise für $\xi$ ergibt sich aus der Menge aller \amm e bezüglich $(S^0,…,S^d)$ mittles $\Pi(\xi)=\nicefrac{S_0^0}{S_1^0}\E_{\mu}(\xi)$
 	\end{itemize}
 	Das legt folgeden Zussammenhang nahe
 	\begin{satz}
 		In jedem \af en\ \epm\ sind äquivalent:
 		\begin{itemize}
 			\item $\Huge=\operatorname{L}^0(\Omega,\F,\P)$, das Modell ist vollständig
 			\item es existiert genau ein \amm
 		\end{itemize}
 	\end{satz}
 	Die Hauptarbeite für den Beweis steckt in folgendem Lemma.
 	\begin{lem}
 		Gegeben sei ein \af es \epm\ $(S^0,…,S^d)$ und ein Kontrakt $\xi\in\L^0(\Omega,\F,\P)$. Setze $Y^i:=\tilde{S}^i_1-\tilde S_0^i$ diskontierter Gewinn mit Basisgut i, $Y=(Y^0,…,Y^d)$. Dann ist 
 		\begin{align*}
 			\pi_{\operatorname{inf}}&=\operatorname{max}\{m\in [-\infty,\infty): \exists x \in \R^{d+1} \text{ mit }m+xY(\omega)\le\nicefrac{S^0_0}{S^0_1}\xi(\omega)\ \P f.s. \}\\
 			\pi_{\operatorname{sup}}&=\operatorname{min}\{m\in (-\infty,\infty]: \exists x \in \R^{d+1} \text{ mit }m+xY(\omega)\ge\nicefrac{S^0_0}{S^0_1}\xi(\omega)\ \P f.s. \}
 		\end{align*}
 		wobei
 		\begin{align*}
 			\pi_{\inf}(\xi)=\inf\Pi(\xi)==\inf\{\text{\af\ Preis für }\xi \}=\inf\{\Ss\E_{\mu}(\xi):\mu\text{ ist \amm},\ \E_{\mu}(|\xi|)<\infty \}
 		\end{align*}
 		mit $m=$ Preisvorschlag, $xY=$ Gewinn mit dem Portfolio $x$, $\Ss\xi(\omega)=$ abdiskontierter Gewinn mit dem Kontrakt.
 	\end{lem}
 	\begin{bemerkung}
 	 	Wir nennen die erste Menge $A_-$ und die zweite $A_+$, jeweils nur die Menge nicht deren Maxima oder Minima.
 	 \end{bemerkung}
 	 Interpretation: \\Als Verkäufer kassiert man den Preis $m$ und verkauft den Kontrakt $\xi$. Beim Marktgeschehen $\omega\in\Omega$ ergibt sich ein Gewinn/Verlust der Höhe $m-\Ss\xi(\omega)$. Falls nun $m\in A_-$, so hätte man auch ein Portfolio $-x$ anlegen können und damit den Gewinn/Verlust $-xY\ge m-\Ss\xi(\omega)$ machen können, egal was $\omega$ ist, daher war der Preis $m$ zu niedrig, da eine \am besteht.\\ \\
 	 Aussage des Lemma: \\ Die Ränder der Mengen $A_-$ und $A_+$ sind Arbitragegrenzen, zwischen welchen der Preis Liegen muss, um nicht zu hoch oder zu niedrig zu sein.
 	 \begin{bew}[Lemma 2.4]
 	 	Wegen $\pi_{\sup}(\xi)=-\pi_{\inf}(-\xi)$ und $A_+(\xi)=-A_-(-\xi)$ reicht es, die Aussage für $\pi_{\sup}$ zu zeigen
 	 	\begin{enumerate}
 	 		\item $\pi_{\sup}(\xi)\le\inf A_+(\xi):$ Sei $m\in A_+$ und $x\in\R^{d+1}$ mit $m+xY\ge\Ss\xi\ \P f.s.$. Für jedes \amm $\mu$ mit $\E(|\xi|)<\infty$ ist, wegen $\mu$ Martingalmaß
 	 		\begin{align*}
 	 			\Ss\E_{\mu}(\xi)\le m+\E_{\mu}(xY)=m+x\E_{\mu}(Y)=m\Longrightarrow \sup\Pi(\xi)\le m\ \forall m\in A_+
 	 		\end{align*}
 	 		Bilden des $\inf$ über $m\in A_+$ liefert die Behauptung
 	 		\item $\pi_{\sup}(\xi)=\inf A_+(\xi)$: Falls $\pi_{\sup}(\xi)=\infty$, dann ist $\pi_{\sup}(\xi)=\inf A_+(\xi)$ nach 1). Sei nun $\pi_{\sup}(\xi)<\infty$ und $m>\pi_{\sup}(\xi)$. Zu Zeigen ist $m\in A_+(\xi)$. Wegen $m\notin\Pi(\xi)$ gibt es im Modell $(S^0,…,S^d,S^{d+1})$ mit $S^{d+1}_0=m,\ S^{d+1}_1=\xi$ eine \am, das heißt 
 	 		\begin{align*}
 	 			\exists(x,x_{d+1})\in \R^{d+1}x\R\intertext{mit}
 	 			xY(\omega)+x_{d+1}(\Ss\xi(\omega)-m)\ge 0\ \P f.s.\label{*1}\\
 	 			\P(xY(\omega)+x_{d+1}(\Ss\xi(\omega)-m)>0)>0
 	 		\end{align*}
 	 		Hierbei ist $x_{d+1}\neq 0$, sonst wäre $(S^0,…,S^d)$ nicht \af: Außerdem ist $x_{d+1}\le 0$. denn für jedes \amm\ $\mu$ im Modell $( )$ mit $\E_{\mu}(|\xi|)<\infty$ gilt
 	 		\begin{align*}
 	 			0\le\ref{*1}\E_{\mu}(xY+x_{d+1}(\Ss\xi-m))=x_{d+1}(\Ss\E_{\mu}(\xi)-m)\Longrightarrow x_{d+1}\le 0
 	 		\end{align*}
 	 		Also ist $x_{d+1}<0$ und mit $z=-\nicefrac{1}{x_{d+1}}x\in\R^{d+1}$ gilt 
 	 		\begin{align*}
 	 			zY+m=-\nicefrac{1}{x_{d+1}}xY+m\ge\ref{*1}\Ss\xi\ \P f.s.\Longrightarrow m\in A_+(\xi)
 	 		\end{align*}
 	 		\item Zu Zeigen bleibt $\inf A_+(\xi)=\operatorname{min}A_+(\xi)$.\\
 	 		Für  $\pi_{\sup}(\xi)=\infty$ ist das klar, da $\infty\in A_+(\xi)$. Für $\pi_{\sup}(\xi)\infty$ sei $m_n\subset A_+$, mit $\lim_{n\to\infty}m_n=\inf A_+=\pi_{\sup}(\xi)$. Zu jedem $m_n$ wähle $x^{(n)}\in\R^{d+1}$ mit $m_n+x^{(n)}Y\ge\Ss\xi\ \P f.s.$. Wegen $y^0=\tilde S^0_1-S^0_0=1-1=0$ kann man $x^{(n)}_0=0$ wählen. Das tun wir.
 	 		\begin{enumerate}
 	 			\item 1.Fall $\operatorname{liminf}_{n\to\infty}|x^{(n)}|<\infty$, das heißt es existiert eine konvergente Teilfolge $x^{(n_k)}$, setze $x:=\lim_{k\to\infty}x^{(n_k)}$. Dann ist $\Ss\xi\le m_{n_k}+x^{n_k}Y\overset{k\to\infty}{\longrightarrow}\pi_{\sup}(\xi)+xY\Longrightarrow \pi_{\sup}(\xi)\in A_+$
 	 			\item 2.Fall $\lim_{n\to\infty}|x^{(n)}|=\infty$. Dann existiert ein $n_0\in\N$ so dann $x^{(n)}>0 $ $\forall n>n_0$. Setze $z^{(n)}=\nicefrac{x^{(n)}}{|x^{(n)}|}$, dann ist $|z^{(n)}|=1$ daraus folgt, dass ein konvergente Teilfolge $z^{(n_k)}$ existiert mit $z:=\lim_{k\to\infty}z^{(n_k)},\ |z|=1,\ z_0=0$
 	 			\begin{align*}
 	 				0+zY=\lim_{k_to\infty}(\nicefrac{m_{n_k}}{|x^{(n_k)}|}+z^{(n_k)}Y)\ge\lim_{k\to\infty}\nicefrac{1}{|x^(n_k)|}\Ss\xi=0
 	 			\end{align*}
 	 			Wegen $(S^0,…,S^d)$ \af impliziert $zY\ge 0\ \P f.s.$ das $zY=0\ \P f.s.$
 	 		\end{enumerate}
 	 		 	 	\end{enumerate}
 	 \end{bew}
\end{document}