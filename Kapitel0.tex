\section{Modellbildung und Einführung}
	Modelliert werden sollen Handelsstrategien für Güter  auf einem Markt, z.B. Staatsanleihen, Aktien, \ldots
	\begin{itemize}
		\item das j-te Handelsgut  zur Zeit $t\geq 0$ hat Preis $S_t^j$
		\item ein Händler hat zur Zeit $t\geq 0$ die Stückzahl $x_t^j\ (\phi_t^j)$ des j-ten Handelsgutes
		\item der Wert des Portfolios zum Zeitpunkt $t\geq 0$ ist \[ x_t\cdot S_t=\sum_{j=1}^nx_t^jS_t^j\]
		\item die Preise $S_t^j$ hängen vom Zufall ab, die Stückzahl $x_t^j$ kann der Händeler durch Kauf und Verkauf beeinflussen
		\item $x_t^j$ muss nicht aus $\N$ sein; $t\mapsto\phi_t$ heißt Handelsstrategie
	\end{itemize}
	\begin{frag}\leavevmode
		\begin{enumerate}
			\item mit welcher Handelsstrategie werde ich sicher Reich?
			\item Was ist der faire Preis einer Option?
		\end{enumerate}
	\end{frag}