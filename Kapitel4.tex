 	\section{Der 2. Hauptsatz der Preistheorie im \epm}
 	Wir wissen:
 	\begin{itemize}
 		\item Wenn der Markt vollständig ist, dann ist nach def 3.11 jeder Kontrakt replizierbar und dies ist äquivalent dazu, dass es für jeden Kontrakt genau einen \af en Preis gibt
 		\item Für $\xi\in\mathcal{H}$ und für jedes Martingalmaß $\mu$ ist $\hat{\Pi}(\xi)=\nicefrac{S_0^0}{S_1^0}\E_{\mu}(\xi)$ nach satz 3.7
 		\item Die Menge aller \af \ Preise für $\xi$ ergibt sich aus der Menge aller \amm e bezüglich $(S^0,…,S^d)$ mittles $\Pi(\xi)=\nicefrac{S_0^0}{S_1^0}\E_{\mu}(\xi)$
 	\end{itemize}
 	Das legt folgeden Zussammenhang nahe
 	\begin{satz}
 		In jedem \af en\ \epm\ sind äquivalent:
 		\begin{itemize}
 			\item $\Huge=\operatorname{L}^0(\Omega,\F,\P)$, das Modell ist vollständig
 			\item es existiert genau ein \amm
 		\end{itemize}
 	\end{satz}
 	Die Hauptarbeite für den Beweis steckt in folgendem Lemma.
 	\begin{lem}
 		Gegeben sei ein \af es \epm\ $(S^0,…,S^d)$ und ein Kontrakt $\xi\in\L^0(\Omega,\F,\P)$. Setze $Y^i:=\tilde{S}^i_1-\tilde S_0^i$ diskontierter Gewinn mit Basisgut i, $Y=(Y^0,…,Y^d)$. Dann ist 
 		\begin{align*}
 			\pi_{\operatorname{inf}}&=\operatorname{max}\{m\in [-\infty,\infty): \exists x \in \R^{d+1} \text{ mit }m+xY(\omega)\le\nicefrac{S^0_0}{S^0_1}\xi(\omega)\ \P f.s. \}\\
 			\pi_{\operatorname{sup}}&=\operatorname{min}\{m\in (-\infty,\infty]: \exists x \in \R^{d+1} \text{ mit }m+xY(\omega)\ge\nicefrac{S^0_0}{S^0_1}\xi(\omega)\ \P f.s. \}
 		\end{align*}
 		wobei
 		\begin{align*}
 			\pi_{\inf}(\xi)=\inf\Pi(\xi)==\inf\{\text{\af\ Preis für }\xi \}=\inf\{\Ss\E_{\mu}(\xi):\mu\text{ ist \amm},\ \E_{\mu}(|\xi|)<\infty \}
 		\end{align*}
 		mit $m=$ Preisvorschlag, $xY=$ Gewinn mit dem Portfolio $x$, $\Ss\xi(\omega)=$ abdiskontierter Gewinn mit dem Kontrakt.
 	\end{lem}
 	\begin{bemerkung}
 	 	Wir nennen die erste Menge $A_-$ und die zweite $A_+$, jeweils nur die Menge nicht deren Maxima oder Minima.
 	 \end{bemerkung}
 	 Interpretation: \\Als Verkäufer kassiert man den Preis $m$ und verkauft den Kontrakt $\xi$. Beim Marktgeschehen $\omega\in\Omega$ ergibt sich ein Gewinn/Verlust der Höhe $m-\Ss\xi(\omega)$. Falls nun $m\in A_-$, so hätte man auch ein Portfolio $-x$ anlegen können und damit den Gewinn/Verlust $-xY\ge m-\Ss\xi(\omega)$ machen können, egal was $\omega$ ist, daher war der Preis $m$ zu niedrig, da eine \am besteht.\\ \\
 	 Aussage des Lemma: \\ Die Ränder der Mengen $A_-$ und $A_+$ sind Arbitragegrenzen, zwischen welchen der Preis Liegen muss, um nicht zu hoch oder zu niedrig zu sein.
 	 \begin{bew}[Lemma 2.4]
 	 	Wegen $\pi_{\sup}(\xi)=-\pi_{\inf}(-\xi)$ und $A_+(\xi)=-A_-(-\xi)$ reicht es, die Aussage für $\pi_{\sup}$ zu zeigen
 	 	\begin{enumerate}
 	 		\item $\pi_{\sup}(\xi)\le\inf A_+(\xi):$ Sei $m\in A_+$ und $x\in\R^{d+1}$ mit $m+xY\ge\Ss\xi\ \P f.s.$. Für jedes \amm $\mu$ mit $\E(|\xi|)<\infty$ ist, wegen $\mu$ Martingalmaß
 	 		\begin{align*}
 	 			\Ss\E_{\mu}(\xi)\le m+\E_{\mu}(xY)=m+x\E_{\mu}(Y)=m\Longrightarrow \sup\Pi(\xi)\le m\ \forall m\in A_+
 	 		\end{align*}
 	 		Bilden des $\inf$ über $m\in A_+$ liefert die Behauptung
 	 		\item $\pi_{\sup}(\xi)=\inf A_+(\xi)$: Falls $\pi_{\sup}(\xi)=\infty$, dann ist $\pi_{\sup}(\xi)=\inf A_+(\xi)$ nach 1). Sei nun $\pi_{\sup}(\xi)<\infty$ und $m>\pi_{\sup}(\xi)$. Zu Zeigen ist $m\in A_+(\xi)$. Wegen $m\notin\Pi(\xi)$ gibt es im Modell $(S^0,…,S^d,S^{d+1})$ mit $S^{d+1}_0=m,\ S^{d+1}_1=\xi$ eine \am, das heißt 
 	 		\begin{align*}
 	 			\exists(x,x_{d+1})\in \R^{d+1}x\R\intertext{mit}
 	 			xY(\omega)+x_{d+1}(\Ss\xi(\omega)-m)\ge 0\ \P f.s.\label{*1}\\
 	 			\P(xY(\omega)+x_{d+1}(\Ss\xi(\omega)-m)>0)>0
 	 		\end{align*}
 	 		Hierbei ist $x_{d+1}\neq 0$, sonst wäre $(S^0,…,S^d)$ nicht \af: Außerdem ist $x_{d+1}\le 0$. denn für jedes \amm\ $\mu$ im Modell $( )$ mit $\E_{\mu}(|\xi|)<\infty$ gilt
 	 		\begin{align*}
 	 			0\le\ref{*1}\E_{\mu}(xY+x_{d+1}(\Ss\xi-m))=x_{d+1}(\Ss\E_{\mu}(\xi)-m)\Longrightarrow x_{d+1}\le 0
 	 		\end{align*}
 	 		Also ist $x_{d+1}<0$ und mit $z=-\nicefrac{1}{x_{d+1}}x\in\R^{d+1}$ gilt 
 	 		\begin{align*}
 	 			zY+m=-\nicefrac{1}{x_{d+1}}xY+m\ge\ref{*1}\Ss\xi\ \P f.s.\Longrightarrow m\in A_+(\xi)
 	 		\end{align*}
 	 		\item Zu Zeigen bleibt $\inf A_+(\xi)=\operatorname{min}A_+(\xi)$.\\
 	 		Für  $\pi_{\sup}(\xi)=\infty$ ist das klar, da $\infty\in A_+(\xi)$. Für $\pi_{\sup}(\xi)\infty$ sei $m_n\subset A_+$, mit $\lim_{n\to\infty}m_n=\inf A_+=\pi_{\sup}(\xi)$. Zu jedem $m_n$ wähle $x^{(n)}\in\R^{d+1}$ mit $m_n+x^{(n)}Y\ge\Ss\xi\ \P f.s.$. Wegen $y^0=\tilde S^0_1-S^0_0=1-1=0$ kann man $x^{(n)}_0=0$ wählen. Das tun wir.
 	 		\begin{enumerate}
 	 			\item 1.Fall $\operatorname{liminf}_{n\to\infty}|x^{(n)}|<\infty$, das heißt es existiert eine konvergente Teilfolge $x^{(n_k)}$, setze $x:=\lim_{k\to\infty}x^{(n_k)}$. Dann ist $\Ss\xi\le m_{n_k}+x^{n_k}Y\overset{k\to\infty}{\longrightarrow}\pi_{\sup}(\xi)+xY\Longrightarrow \pi_{\sup}(\xi)\in A_+$
 	 			\item 2.Fall $\lim_{n\to\infty}|x^{(n)}|=\infty$. Dann existiert ein $n_0\in\N$ so dann $x^{(n)}>0 $ $\forall n>n_0$. Setze $z^{(n)}=\nicefrac{x^{(n)}}{|x^{(n)}|}$, dann ist $|z^{(n)}|=1$ daraus folgt, dass ein konvergente Teilfolge $z^{(n_k)}$ existiert mit $z:=\lim_{k\to\infty}z^{(n_k)},\ |z|=1,\ z_0=0$
 	 			\begin{align*}
 	 				0+zY=\lim_{k_to\infty}(\nicefrac{m_{n_k}}{|x^{(n_k)}|}+z^{(n_k)}Y)\ge\lim_{k\to\infty}\nicefrac{1}{|x^(n_k)|}\Ss\xi=0
 	 			\end{align*}
 	 			Wegen $(S^0,…,S^d)$ \af impliziert $zY\ge 0\ \P f.s.$ das $zY=0\ \P f.s.$
 	 		\end{enumerate}
 	 		 	 	\end{enumerate}
 	 \end{bew}